\documentclass{article}

\newcounter{total}

\newcounter{Cenv}
\newcounter{Citem}[total]
\newcounter{Eenv}
\newcounter{Eitem}[total]

\renewcommand\thetotal{(C/E\arabic{total})}

\renewcommand\theCenv{(C\arabic{Cenv})}
\renewcommand\theCitem{(C\arabic{Cenv}.\roman{Citem})}
\renewcommand\theEenv{(E\arabic{Eenv})}
\renewcommand\theEitem{(E\arabic{Eenv}.\roman{Eitem})}

\newenvironment{Cenv}{
  \setcounter{Cenv}{\value{total}}%
  \stepcounter{total}%
  \refstepcounter{Cenv}%
  \bigskip\par\textbf{Conversation \arabic{Cenv}}%
  \renewcommand\item[1]{%
    \\\refstepcounter{Citem}/C/##1%
  }
}{\par\bigskip}

\newenvironment{Eenv}{
  \setcounter{Eenv}{\value{total}}%
  \stepcounter{total}%
  \refstepcounter{Eenv}%
  \bigskip\par\textbf{Example \arabic{Eenv}}%
  \renewcommand\item[1]{%
    \\\refstepcounter{Eitem}/E/##1%
  }
}{\par\bigskip}

\begin{document}

\section{I'm going to focus on the following references}

\ref{c:1:b} (literally, C1B) and \ref{e:3:a} (should be E3A).

\begin{Cenv}
  \label{c:1}
  \item{Conv 1 line a \label{c:1:a}}
  \item{Conv 1 line b \label{c:1:b}}
  \item{Conv 1 line c \label{c:1:c}}
\end{Cenv}

\noindent
Right, so obviously:

\begin{Cenv}
  \label{c:2}
  \item{Conv 2 line a \label{c:2:a}}
  \item{Conv 2 line b \label{c:2:b}}
\end{Cenv}

\noindent
Now you can see the resulting example...

\begin{Eenv}
  \label{e:3}
  \item{Ex 3 line a \label{e:3:a}}
  \item{Ex 3 line b \label{e:3:b}}
\end{Eenv}

\noindent
And it all comes back to \ref{c:1:c} (should be C1C) and \ref{c:1:a} (should be C1A).

\medskip

\noindent
And in total: \ref{c:1} (i.e, C1), \ref{c:2} (= C2), and \ref{e:3} (= E3).

\end{document}
